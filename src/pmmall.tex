\documentclass[11p]{article}
% Packages
\usepackage{amsmath}
\usepackage{graphicx}
\usepackage[swedish]{babel}
\usepackage[
    backend=biber,
    style=authoryear-ibid,
    sorting=ynt
]{biblatex}
\usepackage[utf8]{inputenc}
\usepackage[T1]{fontenc}
%Källor
\addbibresource{mall.bib}
\graphicspath{ {./images/} }

\title{PMmall \\ \small Fysik 1}
\author{Magnus Silverdal }
\date{\today}

\begin{document}

    \begin{titlepage}
        \begin{center}
            \vspace*{1cm}

            \Huge
            \textbf{Är kärnkraft en bra energimetod?}

            \vspace{0.5cm}

            \vspace{1.5cm}

            \textbf{Tobias Bäckman}

            \vfill

            Ett PM om energiförsörjning \\
            Fysik 1

            \vspace{0.8cm}

            \includegraphics[width=0.4\textwidth]{../images/NTI Gymnasiet_Symbol_print_svart.png}

            \Large
            Teknikprogrammet\\
            NTI Gymnasiet\\
            Umeå\\
            \today

        \end{center}
    \end{titlepage}
% Om arbetet är långt har det en innehållsförteckning, annars kan den utelämnas
    \tableofcontents
    \newpage

    \section{Inledning}
    Elektricitet använder så gott som alla människor i de flesta delar av världen och är en av de viktigaste
    resurser som finns.\ Det är då viktigt att kunna skapa ström så att det räcker till.\ Tyvärr så är det
    inte tillräckligt att bara skapa elen för att vissa former att skapa elektricitet på har stora bieffekter.\ 
    Såsom att elda olja eller kol släpper ut enorma mängder koldioxid och andra dåliga utsläpp.\ 
    Därför ska vi kolla djupare på kärnkraft och om det är ett bra alternativ till dessa dåliga former at elförsörjning. 
    \subsection{frågeställningar}
    %rada upp dina frågor i punktform
    \begin{enumerate}
        \item Hur fungerar kärnkraft?
        \item Hur påverkar kärnkraft miljön?
        \item Är kärnkraft en vettig substitut till fossilbränslen?
    \end{enumerate}

    \section{Resultat}
    \subsection{Kärnkraft, så fungerar det}
    När man eldar kol så får ut värme som fångas upp av vatten som man sedan omvandlar till
    elektricitet genom att snurra på en turbin med vatten eller vattenånga.\ Kärnkraft (specifikt fission som vi pratar om)
    fungerar på samma sätt men istället för att elda kol så använder man kärnklyvning för att få ut energi.\
    Då använder man tunga ämnen, specifikt uranium 235 och skickar in en neutron i atomens kärna.\
    När man gör det så delas kärnan i två som blir nya lättare ämnen och när det händer så släpps det ut en enorm mängd energi.\
    Man klyver kärnan med en neutron och när den delas får man två lättare ämnen, men man får också ut två eller tre lösa neutroner.\
    Det bildar en kedjereaktion och om man inte gör någonting för att hålla reaktionen under kontroll så kommer
    det ske allt snabbare och till slut bli en härdsmälta.\
    Därför behövs det någonting i reaktorn som fångar upp en del av de lösa neutronerna
    så att reaktionen håller sig själv igång men inte går ur kontroll. (Strålsäkerhetsmyndigheten)\
    Då används styrstavar som är gjorda av ett flertal olika ämnen men framförallt borkarbid (B4C)
    som är ett väldigt neutron absorberande ämne.(Wikipedia "styrstav")\
\newpage
    \subsection{Globala miljökonsekvenser av kärnkraft}
    Du har säkert hört talas om chernobyl olyckan år 1986 där reaktor fyra exploderade och ödelagde
    hela staden med flera andra stora konsekvenser över hela europa, den förmodligen största kärnolyckan någonsin.(World-nuclear.org)\
    Du har kanske också hårt talas om fukushima olyckan som är den förmodligen näst största kärnolyckan.(World-nuclear.org)\
    Efter man hört det så är det lätt att säga att kärnkraft borde inte användas tack vare dess stora konsekvenser ifall en olycka sker.\
    Men om man går lite djupare så hittar man att chernobyl olyckan hände på grund av flera lätt förebyggbara saker.\
    Bland annat så var de som jobbade i chernobyl otränade och visste knappt ens vad radioaktivitet var, någonting man
    kan tycka borde vara ganska viktigt för kärnreaktor arbetare att kunna.\ De skulle göra tester på reaktorn och på grund av
    flera otursamma händelser så exploderade reaktorn.\ Reaktorn var också felbyggd från början vilket gjorde så att
    olyckan ens var möjlig.\vspace{0.5cm}
    Fukushima olyckan hade några extrema händelser som orsakade den.\ En enorm jordbävning på magnitud 9.0 slog till på kraftverket vilket
    den stod emot, däremot den femton meter höga tsunamin som kom strax efter slog ut kylningen på några reaktorer i
    fukushima kärnkraftverket.\ Det gjorde så att de var tvungna att släppa ut en del ånga från insidan av en reaktor så att den inte skulle
    bli övertryckt och explodera som den i chernobyl.\ Vattnet från insidan av reaktorn hade samlat upp radioaktiva ämnen vilket den tog
    med sig ut när de var tvungna att sänka trycket i reaktorn.\
    Moderna reaktorer är byggda med betydligt mer säkerhet än den i chernobyl och inte på platser som
    är lika benägna till jordbävningar och tsunamis.\vspace{0.5cm} Moderna reaktorer kan också inte få en härdsmälta på samma sätt
    som chernobyl kunde.\ Reaktorn i chernobyl hade grafit som moderator för att sakta in neutronerna så att de kunde klyva nya atomer.\
    När reaktionen då gick ur kontroll så kunde den fortsätta gå ur kontroll och till slut explodera.\
    Dagens svenska reaktorer har vanligt vatten som moderator så om reaktionen går ur kontroll så släpps det ut en enorm mängd värme.\
    Om vatten hettas upp tillräckligt mycket så kokar det vilket gör så att reaktionen inte längre kan fortsätta och en härdsmälta
    blir effektivt omöjlig att få.\ Nu när du hört det så är det kanske lättare att tänka sig att använda kärnkraft som istället för
    fossila bränslen.\ Kärnkraft har så gott som inga miljökonsekvenser så länge en olycka inte sker, och som jag sa
    så är olyckor som chernobyl och fukushima väldigt låg chans att det händer.\ Den enda biprodukten man får från
    kärnkraft är radioaktivt avfall vilket är bäst förvarat lång under marken i några hundra tusen år.\ Så länge
    det förvaras på rätt sätt så har kärnkraft så gott som inga biprodukter.\
\newpage
    \subsection{Så, kan kärnkraft sustituera fossilbränslen?}
    Svaret på det är ett klart ja och borde bytas in i stället för fossila bränslen på så många ställen som möjligt.\
    Om man eldar olja eller bensin så kan man få ut ungefär 12 kWh energi men det släpper också ut cirka 3kg CO2.\
    Kol ger ungefär 9 kWh om det är av hög kvalitet men bara ungefär hälften om det är av låg kvalitet och det släpper ut 3,7 kg CO2.\
    Metan som är en naturgas ger ungefär 14 kWh men också 2,75 kg CO2.(Lars-kamel)\
    Däremot ett kg uranium innehåller lika mycket energi som 90 TON kol.\ Om vi antar att der är av hög kvalitets kol
    så är det ungefär 810.000 kWh på bara ett kg, och det bästa är att det inte släpper ut ett enda gram av CO2.\
    Det behövs alltså väldigt lite uranium för att få ut en extrem mängd miljövänlig energi.\ Kärnkraft är inte
    ett slutmål mot miljövänlig energi men är ett steg från kol, olja och andra fossilbränslen som borde tas
    för att bromsa den globala uppvärmningen innan vi kan byta till helt miljövänliga energialternativ som sol, vind och fusionskraft.\

    \section{Slutsatser}
    Vi borde byta bort alla bränslen för att skapa elektricitet som släpper ut koldioxid till kärnkraft.\ Efter vi gjort det så kan vi
    börja byta bort kärnkraften mot solkraft, vindkraft och när det är ett fungerande alternativ för att få el, fusionskraft.\

% Mer saker som du kan ha nytta av.

    \section{Referenser}
   % Referenser i text kan skrivas på två sätt: Enligt \textcite{Jens} kan man använde två typer av referenser, inbäddade i texten eller efter ett fakta \parencite{Fraenkel}. Ett till test för att se hur det ser ut \parencite[sid 55]{fermi}.
    https://www.stralsakerhetsmyndigheten.se/omraden/karnkraft/sa-fungerar-ett-karnkraftverk/
    https://sv.wikipedia.org/wiki/Styrstav
    \vspace{0.5cm}
    https://world-nuclear.org/information-library/safety-and-security/safety-of-plants/chernobyl-accident.aspx
    https://world-nuclear.org/information-library/safety-and-security/safety-of-plants/fukushima-daiichi-accident.aspx
    \vspace{0.5cm}
    http://www.lars-kamel.se/Vetenskap/FuelBurning.html
   % \section{Annat som kan vara bra att veta}
   % Om du vill ha kodstil och få med alla tecken kan du använda verbatim. då kan du skriva \verb|abcd!"#| utan problem...

    %Citat skrivs mellan de konstiga symbolerna \verb|``| och \verb|''| för att de ska se bra ut ``se bra ut!''.
    %\subsection{En underrubrik}
    %\subsubsection{En underunderrubrik}
    %\subsection{Ekvationer}
    %Det är lätt att skriva matematik i \LaTeX

    %\begin{equation}
     %   F = G \frac{M m}{r^2}
      %  \label{grav}
    %\end{equation}

    %Ekvation (\ref{grav}) känner ni igen...

    %\subsection{figurer}
    %Bilder placeras enklast på detta sätt. placeringen bestämmer \LaTeX och vi kan bara föreslå (h)är, (t)opp eller (b)otten. Ett utropstecken före tvingar lite mer men inte absolut. I bild \ref{varg} visas en varg
    %\begin{figure}[!h]
    %    \includegraphics[width=0.8\textwidth]{../images/accelerationTime.png}
    %    \caption{Acceleration-tid diagram. Källa: Impuls Fysik 1}
    %    \label{varg}
    %\end{figure}
    %\printbibliography

\end{document}
